\chapter*{References}

Chen, J., Zhou, T., Zhang, Y., Luo, S., Chen, H., Chen, D., Li, C. and Li, W. (2022). The reservoir of latent HIV. Frontiers in Cellular and Infection Microbiology, 12.\\

Collins, K. (2004). Resistance of HIV-infected cells to cytotoxic T lymphocytes. Microbes and Infection, 6(5), pp.494–500.\\

Cronk, Q.C.B. (2009). Evolution in reverse gear: the molecular basis of loss and reversal. Cold Spring Harbor Symposia on Quantitative Biology, [online] 74, pp.259–266.\\

Eberle, J. and Gürtler, L. (2012). HIV types, groups, subtypes and recombinant forms: errors in replication, selection pressure and quasispecies. Intervirology, [online] 55(2), pp.79–83.\\

Fanales-Belasio, E., Raimondo, M., Suligoi, B. and Buttò, S. (2010). HIV virology and pathogenetic mechanisms of infection: a brief overview. Annali dell’Istituto Superiore di Sanità\\

García, M., López-Fernández, L., Mínguez, P., Morón-López, S., Restrepo, C., Navarrete-Muñoz, M.A., López-Bernaldo, J.C., Benguría, A., García, M.I., Cabello, A., Fernández-Guerrero, M., De la Hera, F.J., Estrada, V., Barros, C., Martínez-Picado, J., Górgolas, M., Benito, J.M. and Rallón, N. (2020). Transcriptional signature of resting-memory CD4 T cells differentiates spontaneous from treatment-induced HIV control. Journal of Molecular Medicine, 98(8), pp.1093–1105.\\

Gulzar,     N. and Copeland, K.F.T. (2004). CD8+ T-cells: function and response to HIV infection. Current HIV research, [online] 2(1), pp.23–37.\\

Holmes,       E.C., Nee, S., Rambaut, A., Garnett, G.P. and Harvey, P.H. (1995). Revealing the history of infectious disease epidemics through phylogenetic trees. Philosophical Transactions of the Royal Society of London. Series B, Biological Sciences, [online] 349(1327), pp.33–40. \\

Ijeoma Aniekwu, N. and Atsenuwa, A. (2007). Sexual Violence and HIV/AIDS in Sub-Saharan Africa: An Intimate Link. Local Environment, 12(3), pp.313–324.\\

Jiang, Y., Li, X., Luo, H., Yin, S. and Kaynak, O. (2022). Quo vadis artificial intelligence? Discover Artificial Intelligence, [online] 2(4).\\

Kandathil,         A.J., Ramalingam, S., Kannangai, R., David, S. and Sridharan, G. (2005). Molecular epidemiology of HIV. The Indian Journal of Medical Research, [online] 121(4), pp.333–344.\\

Kanekar, A.S. and Omare, D. (2011). Determinants of HIV/AIDS in armed conflict populations. Journal of Public Health in Africa, [online] 2(1).\\

Khan,  N. and Geiger, J.D. (2021). Role of Viral Protein U (Vpu) in HIV-1 Infection and Pathogenesis. Viruses, 13(8), p.1466.\\

Kogan,             M. and Rappaport, J. (2011). HIV-1 Accessory Protein Vpr: Relevance in the pathogenesis of HIV and potential for therapeutic intervention. Retrovirology, 8(1).\\

Ojha,              K.K., Mishra, S. and Singh, V.K. (2022). Computational molecular phylogeny: concepts and applications. Bioinformatics, pp.67–89.\\

Paraskevis,               D., Pybus, O., Magiorkinis, G., Hatzakis, A., Wensing, A.M., van de Vijver, D.A., Albert, J., Angarano, G., Asjö, B., Balotta, C., Boeri, E., Camacho, R., Chaix, M.-L., Coughlan, S., Costagliola, D., De Luca, A., de Mendoza, C., Derdelinckx, I., Grossman, Z. and Hamouda, O. (2009). Tracing the HIV-1 subtype B mobility in Europe: a phylogeographic approach. Retrovirology, [online] 6(49).\\

Pevsner,               J. (2015). Bioinformatics and functional genomics. Chichester, West Sussex: Wiley Blackwell.\\

Seitz, R. (2016). Human Immunodeficiency Virus (HIV). Transfusion Medicine and Hemotherapy, 43(3), pp.203–222.\\

State               Statistics Committee of Ukraine (2007). Всеукраїнський перепис населення 2001 | English version | Results | General results of the census | National composition of population: [online] web.archive.org. \\


Statista. (2023). Ukrainian refugees by selected country 2022. [online] \\
Available at: https://www.statista.com/statistics/1312584/ukrainian-refugees-by-country/ [Accessed 04 Apr. 2023].\\

Stojsavljevic, J. (1995). Women, conflict, and culture in former Yugoslavia. Gender & Development, 3(1), pp.36–41. \\

Turner,               B.G. and Summers, M.F. (1999). Structural biology of HIV 1 1Edited by P. E. Wright. Journal of Molecular Biology, 285(1), pp.1–32.\\

Vasylyeva, T.I., Liulchuk, M., Friedman, S.R., Sazonova, I., Faria, N.R., Katzourakis, A., Babii, N., Scherbinska, A., Thézé, J., Pybus, O.G., Smyrnov, P., Mbisa, J.L., Paraskevis, D., Hatzakis, A. and Magiorkinis, G. (2018). Molecular epidemiology reveals the role of war in the spread of HIV in Ukraine. Proceedings of the National Academy of Sciences, [online] 115(5).\\

World Health Organization (2021). HIV/AIDS. [online] www.who.int. Available at: https://www.who.int/health-topics/hiv-aids#tab=tab_1.\\

Zhang, H., Pomerantz, R.J., Dornadula, G. and Sun, Y. (2000). Human Immunodeficiency Virus Type 1 Vif Protein Is an Integral Component of an mRNP Complex of Viral RNA and Could Be Involved in the Viral RNA Folding and Packaging Process. Journal of Virology, [online] 74(18), pp.8252–8261.