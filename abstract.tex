\chapter{Abstract}

Bioinformatic analysis is a rapidly developing method of the evaluation of large biological data sets.
The analysis conducted within the bioinformatic field includes a phylogenetic analysis which allows the construction of the phylogenetic trees as a visual representation of the relationship between the viral genome sequences. 
Quantitative analysis such as bootstrapping can be conducted to support the robustness of the phylogenetic tree that was produced during the analysis. \\
One of the viruses that can be evaluated through bioinformatic analysis is HIV, which was studied using this method for many years. 
This research focuses on the subtype B of the virus HIV-1 and evaluates the spread of the virus within the region of Eastern Europe between 2010 and 2015. 
The qualitative analysis of the phylogenetic trees before and after bootstrap analysis showed interesting results as the events of so-called reverse mutations were present within the phylogenetic tree before, but not after bootstrapping was conducted. \\
Following the bootstrap analysis, many of the clades within the tree  gave statistically relevant values i.e. values exceeding 70 (in the range between 0 and 100).  
It was determined by this research that regions of Easter Europe such as countries of former Yugoslavia or countries of former Soviet Union tend to show more statistically relevant transmission events within their group that the other countries of Eastern Europe. 
This analysis refers to past transmission events and does not concern the prediction of potential transmissions based on any statistical model.
