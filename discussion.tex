\chapter{Discussion}
\section{Qualitative Analysis}

The first part of the result section focused mostly on the visual representation of the phylogenetic trees before and after bootstrapping. 
This did not include and statistical analysis of said phylogenetic trees. 
The appearance of both trees was evaluated and was as follows.  

\subsection{Phylogenetic  tree before bootstrapping}

As mentioned in the Results section, Figure 4.1 shows the phylogenetic tree that was not analyzed in MEGA11, with no bootstrap analysis. 
The general analysis of the tree showed some interesting clades were formed between countries that were either close geographically or shared a part of history with each other. \\
One of the clades, the Croatian-Slovenian one, is an excellent example of such relations. 
Both of the countries share a border and were in the past a part of Yugoslavia, which was only dissolved as a country in the mid-90s. 
The same can be said about the Serbian-Slovenian clade, despite not sharing a border, those countries were part of Yugoslavia and it can be assumed that the spread of the virus between those countries can be related both to close the geographical distance between those countries.\\
An analogous situation could be observed within the Estonian clade which contained sequences from Russia, Belarus, and Ukraine within it. 
All four of those countries are geographically close to each other, and all of them have a direct border with Russia. 
Additionally, all four of the countries were part of the USSR until it was dissolved in 1991. 
Due to that reason, frequent migrations within former Yugoslavian countries and the former USSR could be assumed. 
Moreover, countries such as Ukraine and Belarus are under large Russian political influence to this day. 


\subsubsection{Back-Mutations}

Within Figure 4.1, some reverse, or back-mutations were observed. 
As stated by \cite{cronk_2009_evolution} Cronk this kind of reversal to the ancestral state can happen due to many factors such as environmental pressure. 
This issue applied to some Hungarian sequences such as B.HUNGARY.2013.1 which evolutionary reversed from the Slovakian sequence B.SLOVAKIA.2012.1.
Sequence B.HUNGARY.2015.1 which reversed from Ukrainian sequence B.UKRAINE.2012.2, and B.HUNGARY.2014.1 which reversed from another Ukrainian sequence B.UKRAINE.2015.1. 
This issue also applied for one Slovakian sequence B.SLOVAKIA.2011.1 which evolutionary reversed from Hungarian sequence B.HUNGARY.2014.2.
However, with the last two examples, the reverse-mutated sequence was sequenced before the one it was reverting from. 
Interestingly, this issue has not appeared in Figure 4.2 which represented the tree after bootstrap analysis was conducted. 
The back-mutated sequences were reassorted within different clades. 
This strongly suggests that those sequences did not have to be an actual example of reverse mutations. 

\subsection{Phylogenetic tree after bootstrapping}

As mentioned above Figure 4.1 showed many clades that concentrated on the countries of former Yugoslavia. 
As shown in Figure 4.2, some of the sequences from these clades were redistributed into different clades after bootstrap analysis was performed. 
However, most of the time the new clades that were formed after bootstrapping still heavily concentrated on that regions. \\
As an example, the Serbian sequences were redistributed into the Croatian clade which resulted in the formation of the Serbian-Croatian-Slovenian clade. 
The Estonian seems relatively similar before and after bootstrapping. 
A large part of this branch consists of Estonian sequences, with the addition of sequences from Belarus, Russia, and Ukraine. 
This supports the idea that the spread of the virus between countries of the former USSR with each other happens more often than with other non-former USSR countries. \\
There was also an interesting branch containing sequences from Poland and Russia, based on the approximate time when those sequences diverged from each other, it can be speculated that the relationship between those sequences dates back to the times when Poland was under strong Russian political influence.  


\section{Bootstrap values analysis.}

Qualitative analysis of the tree should not be treated as the factual state of the relationship between the sequences, but it provides an insight into which of the areas or clades could be interesting to analyze once bootstrap analysis is performed.
As mentioned before, bootstrapping allows determining how probable it is that the sequences grouped in a specific clade were actually related to each other.
Bootstrap values range from 0-100, and the closer the value to 100, the more probable the relativeness of the sequences is.
Generally, any value above 70 was considered highly probable.
Bootstrap analysis performed for the analysis of the HIV-1 subtype B spread around Eastern Europe gave few high values. 


\subsection{Balkans and Former Yugoslavian countries.}

As seen in Figure 4.3, the highest bootstrap value was obtained between former Yugoslavian countries (Serbia, Croatia, and Slovenia).  
Another set of high values, equal to 99, was obtained between two Croatian sequences and a branch containing Croatian and Slovenian sequences (see Figure 4.4). 
Figure 4.5 provides high values between two Serbian sequences (value = 95), as well as two sets of branches between Slovenian sequences which had a bootstrap value of 95 and 97 accordingly. 
Values over 80 were also seen between Serbian and Slovakian sequences towards the bottom of the figure. \\
Overall, statistically relevant sequences from former Yugoslavia countries were almost always grouped in clades between each other, and not with other Eastern European countries. 
The reason for that can be based on the common history of those countries, as they were all part of one country, Yugoslavia, until very recently. \\
The geopolitical situation of those countries was very complicated which lead to violent conflicts in the late 80s and early 90s. 
As stated by Omare and Kanekar \cite{kanekar_2011_determinants}armed conflict is often a catalyst for the spread of HIV/AIDS or analogically other diseases. 
Additionally, the article draws attention to the increase in HIV infections due to violence against women which happens frequently during wars and military conflicts. 
This conclusion was made by another research done by Aniekwu and Atsenuwa \cite{ijeomaaniekwu_2007_sexual} and referred to violence in Sub-Saharan Africa, however, this could be also applied to the armed conflict in Yugoslavia as war crimes and violence against women was reported in this area during the conflict \cite{stojsavljevic_1995_women}

\subsection{Former USSR countries}

An analogous analysis was conducted for the countries of the former USSR. 
As seen in Figure 4.6, high bootstrap values were reported between individual branches of sequences from Russia, Estonia, and Belarus. 
The first example of such a relationship was between Russian and Estonian sequences which obtained a value of 93, another example, was a branch between Russian and Belarusian sequences with a bootstrap value of 99. 
Towards the bottom of the figure, an example of a small clade of Russian and Belarusian sequences obtained bootstrap values of 89 and 99 within itself. 
Figure 4.7 provides another example of, a bit larger clade between Russian and Belarusian sequences where two of the bootstrap values were equal to 89 and one was equal to 99. \\
Interestingly, none of the Ukrainian sequences obtained statistically relevant bootstrap values with any country of the former USSR, or any of the other countries, despite strong Russian influence or having a large population of ethnic Russians in parts of the country such as Crimea.
According to Ukraine's population census in the largest Crimean city, Sevastopol, the largest ethnic group was determined to be ethnically Russian \cite{statestatisticscommitteeofukraine_2007_}.
This may be the reason that the time range of the sequences used in the project was dated between 2010 and 2015 when Ukraine was significantly less influenced by Russia than Belarus. 


\subsubsection{Poland and Soviet Union}

Another interesting clade that obtained relatively high bootstrap values was Polish - the Russian one. 
The clade consists of two larger branches where one contains all Polish sequences and the other Russian sequences and one Belarusian sequence. 
The divergence between the branches happens much earlier than other analyzed branches It could be speculated that the relationship within the clade dates back to the times before the USSR dissolution when Poland was considered a puppet state of the Soviet Union.

