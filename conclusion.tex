\chapter{Conclusion}

In conclusion, bioinformatic analysis is an excellent method of studying the evolution and past transmission events of the viruses such as HIV. 
It offers both qualitative analyses of the phylogenetic tree structure, as well the qualitative analysis such as bootstrapping which allows the statistical interpretation of the factual relativeness of the obtained sequences. \\
In the case of the HIV-1 subtype B, which is considered the most predominant subtype in developed countries could be evaluated in parts of the world such as Eastern Europe as it was recently opened to frequent migration of the population following the fall of communism. \\
The hypothesis that was established for this project assumed that the most frequent transmission events of the subtype B happened between the countries that shared a common history, such as former Yugoslavia countries (Serbia, Croatia, and Slovenia) or former USSR countries (Russia, Estonia, Belarus, and Ukraine). 
This hypothesis was proven by this research as those regions showed the most statistically relevant transmission events within their groups. 
However, this approach cannot be used as a prediction tool for future transmissions.
