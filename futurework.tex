\chapter{Future Work}

Phylogeny and Phylogenetic analysis are powerful tools that allow insight into the evolution of the viruses (or other organisms) and trace the origins of the transmissions. 
It is a well-developed technique that was appreciated since at least the 1990s and was improving ever since. 
Some interesting discoveries were made using those techniques such as the fact that almost all of the transmissions of HIV in Poland happened due to a single infection, whereas in the other countries analyzed during that research, the transmissions were a result of many sources of transmission. \cite{paraskevis_2009_tracing}
Despite the fact that phylogeny is greatly appreciated for its value in tracing the spread and evolution of various viruses, it is mostly concentrating on past infection events.
Predicting the patterns of the infections of emerging diseases would not be possible using this approach. 

\subsubsection{Potential of Artificial Intelligence in the phylogenetic analysis.}

A revolutionary concept of Artificial Intelligence (AI) is developing rapidly in the 21. Century, especially in recent years. 
Using AI has proven to be a quick and efficient solution in many areas of life, and bioinformatics and phylogeny should not be an exception to that. 
There is great potential for AI machine learning and which has immediate access to data from the entire internet, and it is learning to conduct sensible analysis of this data. \cite{jiang_2022_quo}
Ellison has discussed the epidemiology of COVID-19 in the context of using AI however more research needs to be done to estimate the value of statistical models that AI would be able to create or potential predictions made by Artificial Intelligence about possible transmission routes. \cite{ellison_2020_covid19}  

\subsubsection{Evaluation of the impact of the War in Ukraine on the spread of HIV}

Another interesting topic that is currently being researched regarding HIV transmission, is the impact of the war in Ukraine on the spread of the virus. 
According to the data in the Los Alamos database, the predominant subtype of HIV-1 in Ukraine is currently A6 which was the only sub-type reported in this country in the 90s.
Sub-type B has not been reported until 2008, which may be the result of the change of government system from Communist (as part of the USSR) to a Republic, which resulted in opening borders and an increase in migration from countries where the predominant sub-type circulating was sub-type B. \\
As Ukraine was invaded in February 2021 the increase of people seeking refuge in the neighboring countries increased dramatically. 
According to the website Statista.com as of March 2023, over 1.5M Ukrainians found refuge in just Poland \cite{ukrainian}.
There is research discussing the risk of the increase in the spread of sub-type A in the displaced civilians within the country \cite{vasylyeva_2018_molecular}
This suggests that the transmission of HIV can extend to other countries which gave refuge to Ukrainians and potentially, phylogenetic analysis of the spread of HIV in this part of the continent will reflect on the current geopolitical situation. 
